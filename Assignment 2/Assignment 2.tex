\documentclass[journal,12pt,twocolumn]{IEEEtran}

\usepackage{setspace}
\usepackage{gensymb}
\singlespacing
\usepackage[cmex10]{amsmath}

\usepackage{amsthm}

\usepackage{mathrsfs}
\usepackage{txfonts}
\usepackage{stfloats}
\usepackage{bm}
\usepackage{cite}
\usepackage{cases}
\usepackage{subfig}
\usepackage{tasks}

\usepackage{longtable}
\usepackage{multirow}

\usepackage{enumitem}
\usepackage{mathtools}
\usepackage{steinmetz}
\usepackage{tikz}
\usepackage{circuitikz}
\usepackage{verbatim}
\usepackage{tfrupee}
\usepackage[breaklinks=true]{hyperref}
\usepackage{graphicx}
\usepackage{tkz-euclide}

\usetikzlibrary{calc,math}
\usepackage{listings}
    \usepackage{color}                                            %%
    \usepackage{array}                                            %%
    \usepackage{longtable}                                        %%
    \usepackage{calc}                                             %%
    \usepackage{multirow}                                         %%
    \usepackage{hhline}                                           %%
    \usepackage{ifthen}                                           %%
    \usepackage{lscape}     
\usepackage{multicol}
\usepackage{chngcntr}

\DeclareMathOperator*{\Res}{Res}

\renewcommand\thesection{\arabic{section}}
\renewcommand\thesubsection{\thesection.\arabic{subsection}}
\renewcommand\thesubsubsection{\thesubsection.\arabic{subsubsection}}

\renewcommand\thesectiondis{\arabic{section}}
\renewcommand\thesubsectiondis{\thesectiondis.\arabic{subsection}}
\renewcommand\thesubsubsectiondis{\thesubsectiondis.\arabic{sub subsection}}


\hyphenation{optical networks semiconduc-tor}
\def\inputGnumericTable{}                                 %%

\lstset{
%language=C,
frame=single, 
breaklines=true,
columns=fullflexible
}
\date{March 2021}

\begin{document}

\newcommand{\BEQA}{\begin{eqnarray}}
\newcommand{\EEQA}{\end{eqnarray}}
\newcommand{\define}{\stackrel{\triangle}{=}}
\bibliographystyle{IEEEtran}
\raggedbottom
\setlength{\parindent}{0pt}
\providecommand{\mbf}{\mathbf}
\providecommand{\pr}[1]{\ensuremath{\Pr\left(#1\right)}}
\providecommand{\qfunc}[1]{\ensuremath{Q\left(#1\right)}}
\providecommand{\fn}[1]{\ensuremath{f\left(#1\right)}}
\providecommand{\e}[1]{\ensuremath{E\left(#1\right)}}
\providecommand{\sbrak}[1]{\ensuremath{{}\left[#1\right]}}
\providecommand{\lsbrak}[1]{\ensuremath{{}\left[#1\right.}}
\providecommand{\rsbrak}[1]{\ensuremath{{}\left.#1\right]}}
\providecommand{\brak}[1]{\ensuremath{\left(#1\right)}}
\providecommand{\lbrak}[1]{\ensuremath{\left(#1\right.}}
\providecommand{\rbrak}[1]{\ensuremath{\left.#1\right)}}
\providecommand{\cbrak}[1]{\ensuremath{\left\{#1\right\}}}
\providecommand{\lcbrak}[1]{\ensuremath{\left\{#1\right.}}
\providecommand{\rcbrak}[1]{\ensuremath{\left.#1\right\}}}
\theoremstyle{remark}
\newtheorem{rem}{Remark}
\newcommand{\sgn}{\mathop{\mathrm{sgn}}}
\providecommand{\abs}[1]{\vert#1\vert}
\providecommand{\res}[1]{\Res\displaylimits_{#1}} 
\providecommand{\norm}[1]{\lVert#1\rVert}
%\providecommand{\norm}[1]{\lVert#1\rVert}
\providecommand{\mtx}[1]{\mathbf{#1}}
\providecommand{\mean}[1]{E[ #1 ]}
\providecommand{\fourier}{\overset{\mathcal{F}}{ \rightleftharpoons}}
%\providecommand{\hilbert}{\overset{\mathcal{H}}{ \rightleftharpoons}}
\providecommand{\system}{\overset{\mathcal{H}}{ \longleftrightarrow}}
	%\newcommand{\solution}[2]{\textbf{Solution:}{#1}}
\newcommand{\solution}{\noindent \textbf{Solution: }}
\newcommand{\Int}{\int\limits}
\newcommand{\cosec}{\,\text{cosec}\,}
\providecommand{\dec}[2]{\ensuremath{\overset{#1}{\underset{#2}{\gtrless}}}}
\newcommand{\myvec}[1]{\ensuremath{\begin{pmatrix}#1\end{pmatrix}}}
\newcommand{\mydet}[1]{\ensuremath{\begin{vmatrix}#1\end{vmatrix}}}
\numberwithin{equation}{subsection}
\makeatletter
\vspace{3cm}
\title{Assignment 2}
\author{Suraj - CS20BTECH11050}
\maketitle
\newpage
\bigskip
\renewcommand{\thetable}{\theenumi}

and latex-tikz codes from 
%
\begin{lstlisting}
LATEX CODE
\end{lstlisting}

\section{GATE PROBLEM 78} 
The joint probability density function of two random variables X and Y is given as
\[
  f(x,y) =
  \begin{cases}
      \dfrac{6}{5}\brak{x + y^{2}}  & 0\leq x \leq 1\;\;0\leq y \leq 1 \\ \\
                  0                 & \text{elsewhere} 
  \end{cases}
\] 

\begin{tasks}(2)
\task $\dfrac{2}{5}\;\text{and}\;\dfrac{3}{5}$
\task $\dfrac{3}{5}\;\text{and}\;\dfrac{3}{5}$
\task $\dfrac{3}{5}\;\text{and}\;\dfrac{6}{5}$
\task $\dfrac{4}{5}\;\text{and}\;\dfrac{6}{5}$
\end{tasks}

\bigskip

\section{SOLUTION}
For a continuous joint probability distribution  $\e{X}$ \\
and $\e{Y}$ are obtained using the following equations  \\
\eqref{a} and \eqref{b}

\begin{align}
\e{X} &= \Int_{-\infty}^{+\infty}\Int_{-\infty}^{+\infty} x \cdot \fn{x,y}\,dx\,dy \label{a} \\ 
\e{Y} &= \Int_{-\infty}^{+\infty}\Int_{-\infty}^{+\infty} y \cdot \fn{x,y}\,dx\,dy \label{b}
\end{align}
Using equation \eqref{a} \e{X} is calculated as 
\newpage
\begin{align*}
\e{X} &= \Int_{0}^{1}\Int_{0}^{1} x\,\dfrac{6}{5}\brak{x+y^2}\,dx\,dy \;+ 0 \\ 
      &= \Int_{0}^{1}\dfrac{6}{5}\brak{\Int_{0}^{1}x^2\,dx}+\dfrac{6}{5}\,y^2\,\brak{\Int_{0}^{1}x\,dx}\;dy   \\
      &= \Int_{0}^{1}\dfrac{6}{5}\brak{\dfrac{1}{3}}+\dfrac{6}{5}\,y^2\,\brak{\dfrac{1}{2}}\;dy \\
      &= \dfrac{2}{5}\Int_{0}^{1}\,dy + \dfrac{3}{5}\Int_{0}^{1}y^2\,dy  \\
      &= \dfrac{2}{5} + \dfrac{3}{5}\brak{\dfrac{1}{3}} \\
\e{X} &=  \dfrac{3}{5}  
\end{align*}
Using equation \eqref{b} \e{Y} is calculated as
\begin{align*}
\e{Y} &= \Int_{0}^{1}\Int_{0}^{1} y\,\dfrac{6}{5}\brak{x+y^2}\,dx\,dy \;+ 0 \\ 
      &= \Int_{0}^{1}\dfrac{6}{5}\,x\brak{\Int_{0}^{1}y\,dy} + \dfrac{6}{5}\brak{\Int_{0}^{1}y^{3}\,dy}\;dx \; \\ 
      &= \Int_{0}^{1}\dfrac{6}{5}\,x\brak{\dfrac{1}{2}} + \dfrac{6}{5}\brak{\dfrac{1}{4}}\;dx \; \\ 
      &= \dfrac{3}{5}\Int_{0}^{1}x\,dx\;+\;\dfrac{3}{10}\Int_{0}^{1}\,dx  \\
      &= \dfrac{3}{5} \brak{\dfrac{1}{2}} + \dfrac{3}{10} \\
\e{Y} &= \dfrac{3}{5}      
\end{align*}
 $$\therefore \e{X} = \dfrac{3}{5}\;\text{and}\;\e{Y} = \dfrac{3}{5}$$ 
 Hence the answer is \textbf{option b}
\end{document}
