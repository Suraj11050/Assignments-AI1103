\documentclass[journal,12pt,twocolumn]{IEEEtran}

\usepackage{setspace}
\usepackage{gensymb}
\singlespacing
\usepackage[cmex10]{amsmath}

\usepackage{amsthm}

\usepackage{mathrsfs}
\usepackage{txfonts}
\usepackage{stfloats}
\usepackage{bm}
\usepackage{cite}
\usepackage{cases}
\usepackage{subfig}
\usepackage{tasks}

\usepackage{longtable}
\usepackage{multirow}

\usepackage{enumerate}
\usepackage{mathtools}
\usepackage{steinmetz}
\usepackage{tikz}
\usepackage{circuitikz}
\usepackage{verbatim}
\usepackage{tfrupee}
\usepackage[breaklinks=true]{hyperref}
\usepackage{graphicx}
\usepackage{tkz-euclide}

\usetikzlibrary{calc,math}
\usepackage{listings}
    \usepackage{color}                                            %%
    \usepackage{array}                                            %%
    \usepackage{longtable}                                        %%
    \usepackage{calc}                                             %%
    \usepackage{multirow}                                         %%
    \usepackage{hhline}                                           %%
    \usepackage{ifthen}                                           %%
    \usepackage{lscape}     
\usepackage{multicol}
\usepackage{chngcntr}

\DeclareMathOperator*{\Res}{Res}

\renewcommand\thesection{\arabic{section}}
\renewcommand\thesubsection{\thesection.\arabic{subsection}}
\renewcommand\thesubsubsection{\thesubsection.\arabic{subsubsection}}

\renewcommand\thesectiondis{\arabic{section}}
\renewcommand\thesubsectiondis{\thesectiondis.\arabic{subsection}}
\renewcommand\thesubsubsectiondis{\thesubsectiondis.\arabic{sub subsection}}


\hyphenation{optical networks semiconduc-tor}
\def\inputGnumericTable{}                                 %%

\lstset{
%language=C,
frame=single, 
breaklines=true,
columns=fullflexible
}
\date{March 2021}

\begin{document}

\newcommand{\BEQA}{\begin{eqnarray}}
\newcommand{\EEQA}{\end{eqnarray}}
\newcommand{\define}{\stackrel{\triangle}{=}}
\bibliographystyle{IEEEtran}
\raggedbottom
\setlength{\parindent}{0pt}
\providecommand{\mbf}{\mathbf}
\providecommand{\pr}[1]{\ensuremath{\Pr\left(#1\right)}}
\providecommand{\qfunc}[1]{\ensuremath{Q\left(#1\right)}}
\providecommand{\fn}[1]{\ensuremath{f\left(#1\right)}}
\providecommand{\e}[1]{\ensuremath{E\left(#1\right)}}
\providecommand{\sbrak}[1]{\ensuremath{{}\left[#1\right]}}
\providecommand{\lsbrak}[1]{\ensuremath{{}\left[#1\right.}}
\providecommand{\rsbrak}[1]{\ensuremath{{}\left.#1\right]}}
\providecommand{\brak}[1]{\ensuremath{\left(#1\right)}}
\providecommand{\lbrak}[1]{\ensuremath{\left(#1\right.}}
\providecommand{\rbrak}[1]{\ensuremath{\left.#1\right)}}
\providecommand{\cbrak}[1]{\ensuremath{\left\{#1\right\}}}
\providecommand{\lcbrak}[1]{\ensuremath{\left\{#1\right.}}
\providecommand{\rcbrak}[1]{\ensuremath{\left.#1\right\}}}
\theoremstyle{remark}
\newtheorem{rem}{Remark}
\newcommand{\sgn}{\mathop{\mathrm{sgn}}}
\providecommand{\abs}[1]{\vert#1\vert}
\providecommand{\res}[1]{\Res\displaylimits_{#1}} 
\providecommand{\norm}[1]{\lVert#1\rVert}
%\providecommand{\norm}[1]{\lVert#1\rVert}
\providecommand{\mtx}[1]{\mathbf{#1}}
\providecommand{\mean}[1]{E[ #1 ]}
\providecommand{\fourier}{\overset{\mathcal{F}}{ \rightleftharpoons}}
%\providecommand{\hilbert}{\overset{\mathcal{H}}{ \rightleftharpoons}}
\providecommand{\system}{\overset{\mathcal{H}}{ \longleftrightarrow}}
	%\newcommand{\solution}[2]{\textbf{Solution:}{#1}}
\newcommand{\solution}{\noindent \textbf{Solution: }}
\newcommand{\cosec}{\,\text{cosec}\,}
\newcommand{\comp}{\mathsf{c}}
\providecommand{\dec}[2]{\ensuremath{\overset{#1}{\underset{#2}{\gtrless}}}}
\newcommand{\myvec}[1]{\ensuremath{\begin{pmatrix}#1\end{pmatrix}}}
\newcommand{\mydet}[1]{\ensuremath{\begin{vmatrix}#1\end{vmatrix}}}
\numberwithin{equation}{subsection}
\makeatletter
\vspace{3cm}
\title{Assignment 3}
\author{Suraj - CS20BTECH11050}
\maketitle
\newpage
\bigskip
\renewcommand{\thetable}{\theenumi}
Download all python codes from 
\begin{lstlisting}
https://github.com/Suraj11050/Assignments-AI1103/blob/main/Assignment%203/Assignment3.py
\end{lstlisting}
%
Download Latex-tikz codes from 
%
\begin{lstlisting}
https://github.com/Suraj11050/Assignments-AI1103/blob/main/Assignment%203/Assignment3.tex
\end{lstlisting}

\section{GATE 2009 (MA) PROBLEM 16} 
Let F, G and H be pair wise independent events such that $\pr{F}=\pr{G}=\pr{H}=\dfrac{1}{3}$ 
and $\pr{F \cap G \cap H}=\dfrac{1}{4}$ Then the probability that at least one event among F, G and H occurs is 
\begin{enumerate}[(A)]
\begin{multicols}{4}
\setlength\itemsep{2em}
\item $\dfrac{11}{12}$
\item $\dfrac{7}{12}$
\item $\dfrac{5}{12}$
\item $\dfrac{3}{4}$
\end{multicols}
\end{enumerate}

\section{SOLUTION}
If two Events $X_1$ and $X_2$ are independent then 
\begin{equation}
\pr{X_1 X_2} = \pr{X_1} \times \pr{X_2} \label{a}
\end{equation}
Using equation \eqref{a} we get the following results 
\begin{align}
\pr{F G} &= \dfrac{1}{3} \times \dfrac{1}{3} = \dfrac{1}{9}\label{b} \\
\pr{G H} &= \dfrac{1}{3} \times \dfrac{1}{3} = \dfrac{1}{9}\label{c} \\
\pr{H F} &= \dfrac{1}{3} \times \dfrac{1}{3} = \dfrac{1}{9}\label{d}
\end{align}

At least one event among F, G, H should occur is $\pr{F + G + H}$ 
from Principal of inclusion and exclusion it is calculated using random variable as
\begin{multline}
\pr{F+G+H} = \pr{F} + \pr{G} + \pr{H} \\
- \brak{\pr{F G} + \pr{G H} + \pr{H F}} + \pr{F G H} \label{e}
\end{multline}
Substituting above results from equation \eqref{b}, \eqref{c}, \eqref{d} in equation \eqref{e}
\begin{align*}
\pr{F+G+H} &= 3\brak{\dfrac{1}{3}} - 3\brak{\dfrac{1}{9}} + \dfrac{1}{4} \\
\therefore \pr{F+G+H} &= \dfrac{11}{12}
\end{align*}

Hence Probability that at least one event among F, G, H occurs is $\pr{F + G + H}=\dfrac{11}{12}$ and correct answer is \textbf{Option (A)}
\null \par \null
But we know that 
\begin{align}
(F G) &= (F G H) + (F G H^{\comp}) \\
\pr{F G} &= \pr{F G H} + \pr{F G H^{\comp}} \\
\therefore \pr{F G} &\geq \pr{F G H} 
\end{align}
In the given question 
\begin{align}
\pr{F G H} &= \dfrac{1}{4} \\
\pr{F G}   &= \dfrac{1}{9} \\
\pr{F G } &< \pr{F G H}
\end{align}
Which is not possible

Similar case with $\pr{G H}$ and $\pr{H F}$ 

Some of the probabilities turnout to be negative like 
\begin{align}
\pr{F^{\comp} G H}  &= \dfrac{1}{9} - \dfrac{1}{4} = -\,\dfrac{5}{36} \\
\pr{F G^{\comp} H}  &= \dfrac{1}{9} - \dfrac{1}{4} = -\,\dfrac{5}{36} \\
\pr{F^{\comp} G H}  &= \dfrac{1}{9} - \dfrac{1}{4} = -\,\dfrac{5}{36} 
\end{align}
Probability $P\in [0,1]$ but because of the data in the question some of the probabilities turn out to be negative. Therefore \textbf{Question is incorrect}
\end{document}
